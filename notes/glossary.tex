\begin{itemize}[leftmargin=*]
	\item
	\textbf{auto-merge}:
		A remote change to a file that can be automatically combined with your version, merging those changes in.
	\item
		\textbf{branch}:
		A single development stream, or split in the project, allowing you to try things out or develop features without affecting other work.
	\item
		\textbf{checkout}:
		Switch your project directory to a certain version of the project, replaces version managed files with the versions from this point in time.
	\item
		\textbf{conflict}
		A situation where there are overlapping changes in a file, this situation Git will warn you and you’ll need to fix the conflict yourself.
	\item
		\textbf{commit}:
		Create a point in time snapshot/version of the current state of the project files.
	\item
        \textbf{fetch}:
		Fetching file versions and information from central repository server, e.g. GitHub
	\item
        \textbf{the HEAD}:
		The active commit you are currently working from.
	\item
		\textbf{merge}:
		Merging two branches together, for example, your experimental branch back into the normal working branch, to release that experimental code. Applies the changes of the other branch into your current working branch.
    \item
        \textbf{repository / repo}:
        Project files and a versioning database, in our example this is hosted on GitHub, but can be hosted on any Git server or your local machine.
    \item
    	\textbf{pull}
    	Pulling down from the central project repository and updating the branch you are working on.
    \item
    	\textbf{push}
    	Push your snapshots (work), to the central project repository, to allow other people to pull and checkout your changes.
\end{itemize}