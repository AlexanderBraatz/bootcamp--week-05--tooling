\section{Is understanding version management important?}

Building and maintaining code can be a complicated process, especially when you have a complex code base and/or many people involved.
\\

Knowing how to version manage a code base is an essential skillset for any professional developer. 
\\

\textbf{It's worth the learning curve, trust us on this one!}


\section{Why is version management an important skillset?}
\subsection{The nature of developing solutions with code}

Good code products are usually created \textbf{iteratively}. This means you tend to make many small improvements that layer on top of one another. A bit like an onion.

\pagebreak

\begin{infobox}{Iterative and incremental development}
	\quotepage{The basic idea behind this method is to develop a system through repeated cycles (iterative) and in smaller portions at a time (incremental), allowing software developers to take advantage of what was learned during development of earlier parts or versions of the system.}{Wikipedia}
\end{infobox}

The part about building on what you have learned is really important. Coding is always a learning process, even when you are at senior level. It is normal to \textbf{get things wrong a few times before you get it right}, and in order to do that you often need to experiment with solutions.
\\

And once built, code very rarely stands still. It will need continuous attention to manage it. Such as:

\begin{itemize}
    \item Adding new features to respond to customer needs
    \item Fixing bugs
    \item Improving performance and efficiency
    \item Maintaining security
\end{itemize}

These tasks are likely to involve changes across lots of files, and we'll usually want to keep track of:

\begin{itemize}
    \item What files have been changed
    \item When files were changed
    \item Why files were changed (ew feature, bug fix etc)
\end{itemize}

When working in teams, it's usually helpful to also know:

\begin{itemize}
    \item Who created what code
    \item Who reviewed, approved and dealt with code conflicts
    \item What future changes are being worked on
\end{itemize}

There's a lot to keep on top of isn't there? Often a lot of this will be happening at the same time as well.
\\

A strong approach to version management can help make all this stuff a whole bunch easier and ultimately allow you to spend more time coding and less on admin.

\subsection{Approaches to version managing code}

It is possible to version manage your code without tools, using manual approaches. But manual approaches are error prone and time consuming - boo!
\\ 

Approaches to manual version management may include some or all of these:

\begin{itemize}
    \item Duplicating specific files and renaming them
    \item Copying a whole folder or project and renaming it with a timestamp, eg taking a backup
    \item Creating a changelog for each file
\end{itemize}

But we're developers. And developers specialise in using and creating tools to make things quicker and easier. So let's look at what tools we can make use of. 



\section{Introducing Git?}

Git is a software tool that automates much of the version management process for us. It can:

\begin{itemize}
    \item keep track of how files change over time (our code)
    \item can be used to version manage anything (text files, video, images)
    \item mainly used on command line, but GUIs are available
\end{itemize}


\subsection{Why Git?}

There are many version management tools out there, why do we use Git?
\\

Each has its own quirks, and merits in its design.
\\

Ultimately we use Git because:

\begin{itemize}
    \item it is the most widely used source code management tool
	\item 33.3\% of professional software developers reporting use Git or GitHub as their primary source control system
	\item excellent tooling support, supporting existing workflows and editors like VSCode, Sublime
\end{itemize}


\section{Additional resources}

\begin{itemize}[leftmargin=*]
    \item \href{https://en.wikipedia.org/wiki/Iterative_and_incremental_development}{Wikipedia - iterative and incremental development }
\end{itemize}