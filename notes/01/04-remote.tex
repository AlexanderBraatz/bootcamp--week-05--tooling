\section{Creating a new repository}

There can only be one first commit, so consider carefully how it is created.

\subsection{Option 1: first commit on GitHub}

\begin{enumerate}
    \item create your project repo on GitHub
    \item choose to "Initialize this repository with a README" \textbf{this is the first commit, adding this file on the remote}
    \item locally run \texttt{git init}
    \item connect with GitHub with \texttt{git remote add origin {repository URL}}
    \item \texttt{git fetch} to get information about the remote repo
    \item \texttt{git checkout master} to checkout the first commit from GitHub (including your README.md)
    \item start tracking new files with \texttt{git add {filename}}
    \item commiting with \texttt{git commit}
    \item push up your work to GitHub with \texttt{git push}
\end{enumerate}

\subsection{Option 2: first commit on your computer}

\begin{enumerate}
    \item create your project repo on GitHub
    \item DO NOT choose to "Initialize this repository with a README"
    \item locally run \texttt{git init}
    \item connect with GitHub with \texttt{git remote add origin {repository URL}}
    \item \texttt{git add {filename}}
    \item \texttt{git commit -m "adding my first file"} \textbf{this is the first commit, adding this file on the local}
    \item \texttt{git push origin master} first commit makes it from local to the remote
\end{enumerate}

\section{Fewer commands with \texttt{git clone}}

\texttt{git clone \{repository URL\} \{folder to create\}}

A great shortcut for setting up a new project folder and connecting with the remote.
\\

It is equivalent to running all these commands:

\begin{minted}{bash}
    mkdir {folder to create}
    cd {folder to create}
    git init
    git remote add origin {repository URL}
    git fetch
    git checkout master
\end{minted}

\section{Pushing a branch to remote repository}

When working with branches locally (common to a feature branching workflow) you'll often want to get those branches onto the remote.

\begin{minted}{bash}
    git checkout master # optionally going back to master branch at the beginning
    git branch my-feature-branch # create your feature branch
    git checkout my-feature-branch # switch to working on that branch
    # [do work] = make your code changes
    git commit -am "commit message" # commit onto your feature branch
    git push origin my-feature-branch # create branch on the remote and push commits up to it
\end{minted}

git branch --set-upstream-to=origin/master master


Pull requests


cleanup

delete from remote (GitHub)
git checkout master and pull
git branch -d my-feature-branch
git branch -a
git remote prune origin


\begin{infobox}{\texttt{git fetch} vs. \texttt{git pull}}
    What is the difference?

    \texttt{git pull} is in effect a \texttt{git fetch} then a \texttt{git checkout}, going to the remote and fetching any changes, then applying (or checking out) those changes to your version of the files.
\end{infobox}