\newpage

\section*{How To Use This Document}

Bits of text in \textcolor{dm-red}{red} are links and should be clicked at every opportunity. Bits of text in \texttt{monospaced green} represent code. Most the other text is just text: you should probably read it.
\\

Copying and pasting code from a PDF can mess up indentation. For this reason large blocks of code will usually have a [\textcolor{dm-red}{View on GitHub}] link underneath them. If you want to copy and paste the code you should follow the link and copy the file from GitHub.


\subsection*{Taking Notes}

In earlier cohorts I experimented with giving out notes in an editable format. But I found that people would often unintentionally change the notes, which meant that the notes were then wrong. I've switched to using PDFs as they allow for the nicest formatting and are also immune from accidental changes.
\\

Make sure you open the PDF in a PDF viewing app. If you open it in an app that converts it into some other format (e.g. Google Docs) you may well miss out on important formatting, which will make the notes harder to follow.
\\

I make an effort to include all the necessary information in the notes, so you shouldn't need to take any additional notes. However, I know that this doesn't work for everyone. There are various tools that you can use to annotate PDFs:

\begin{itemize}
    \item Preview (Mac)
    \item Edge (Windows)
    \item \href{https://web.hypothes.is/quick-start-guide-for-students/}{Hypothes.is}
    \item Google Drive (\textit{not} Google Docs)
    \item Dropbox
\end{itemize}

\textbf{Do not use a word processor to take programming notes!} (e.g. Google Docs, Word, Pages). Word processors have the nasty habit of converting double-quotes into ``smart-quotes''. These can be almost impossible to spot in a text-editor, but will completely break your code.
